% #############################################################################
% RESUMO em Português
% !TEX root = ../main.tex
% #############################################################################
% use \noindent in firts paragraph

%The present work formalizes and addresses the Flying Tourist Problem (FTP), a NP-hard problem that occurs as a generalization of the Traveling Salesman Problem (TSP), and whose goal is to find the best schedule, route, and set of flights, for any given unconstrained multi-city flight request. In fact, despite the current existence of numerous flight search applications, most of them lack the ability to  properly address unconstrained multi-city flight requests, since this problem is generally not tractable. In accordance, the main goal of this research is to develop a methodology that allows an efficient resolution  of this rather demanding problem. To accomplish this, different heuristics and meta-heuristic optimization algorithms were considered, including the Ant Colony Optimization and the Simulated Annealing, allowing the identification of solutions in real-time, even for large instances. 
%The developed methods were integrated into a web application prototype, allowing a fast resolution of user-defined requests. In particular, the implemented system was evaluated using different criteria, including the quality of its optimization system; the utility of the devised problem; and its performance compared to other similar systems. 
%The obtained results show that the developed optimization system consistently presents solutions that are as close as 10\% to the optimal value and the considered meta-heuristic optimization strategies present solutions that are up to 35\% cheaper than those developed by simpler heuristics. Furthermore, when comparing the developed system to  the only known (but not-disclosed) alternative, it was shown that the developed application provides the cheapest and the best-recommended solutions, respectively 95\% and 74\% of the times. As a result, upon  the planning of a complex multi-city trip, the developed system showed to allow the user to save a significant amount of time and money.

\noindent O presente trabalho formaliza e aborda um problema a que se designou de 
“Flying Tourist Problem” (FTP), pertencente à classe de problemas 
Np-completos, que ocorre como uma generalização do conhecido Problema do 
Caixeiro Viajante, e cujo objectivo é determinar o melhor agendamento, 
rota e conjunto de voos que permitem cumprir um itinerário que passa por 
várias cidades, sem restrições, e realizada apenas com base em voos 
comerciais. Apesar de existirem várias aplicações para suporte à procura 
de voos comerciais, a maioria não possui as ferramentas necessárias para 
abordar o problema em questão, dado o crescimento exponencial da 
complexidade do mesmo. Assim, o principal objectivo deste trabalho é o 
desenvolvimento de uma metodologia eficiente para a resolução deste 
problema. Para concretizar este objectivo, considerou-se a utilização de 
vários algoritmos de optimização heurísticos e meta-heurísticos, 
incluindo os algoritmos de optimização baseados em técnicas de “Ant 
Colony Optimization” (ACO) e de “Simulated Annealing” (SA), o que 
permite a determinação de soluções em tempo real, mesmo para problemas 
de grande dimensão. Os métodos desenvolvidos foram integrados e 
prototipados numa aplicação, permitindo a resolução de problemas reais 
definidos pelo utilizador. Em particular, o sistema implementado foi 
avaliado usando diferentes critérios, incluindo a qualidade do seu 
sistema de optimização; a utilidade do problema proposto; e o seu 
desempenho quando comparado com outros sistemas semelhantes. Os 
resultados obtidos mostram que o sistema de optimização desenvolvido é 
capaz de apresentar soluções cujo erro relativo é inferior a 10\%, e que 
os algoritmos de optimização meta-heurísticos implementados apresentam 
soluções que são até 35\% menos dispendiosas que aquelas produzidas por 
técnicas heurísticas. Além disso, ao comparar o sistema desenvolvido com 
a única alternativa (não aberta) actualmente existente, verificou-se que 
em 95\% das vezes a solução encontrada é a mais barata e que em 74\% das 
vezes corresponde à melhor solução recomendada. Consequentemente, 
mostrou-se que o sistema  desenvolvido oferece vantagens significativas 
para apoio no planeamento de viagens envolvendo várias cidades, 
permitindo aos utilizadores pouparem quantidades significativas de tempo 
e dinheiro.