Every CSA request may be defined as a FTP instance, as proposed in section \ref{sec:ftp}. Thus, each request is described by a limited number of attributes (origin, start date, etc.),  and it may be uniquely identified according to its uniform resource identifier (URI). 
%In order to uniquely identify a particular resource, 
%This means that the identification of a specific resource can be achieved by including each of these attributes in the uniform resource identifier.

Hence, a flight resource is described by a predefined list of pairs of keywords and values. This is illustrated in table  \ref{table:api_symbols}, where the first and second column represent a FTP variable name and symbol, respectively. In its turn, the third column defines the keyword for the URI, while the fourth column provides a description about the acceptable values for each keyword. 

It should be noted that the defined pairs of keywords and values must undergo a validation process. This is to, first, verify if the request is a valid FTP formulation, and second, if the values provided for each keyword are acceptable.  

%A flight resource may be described by an HTTP request to the API endpoint, together with the specification of the resource type ('/flights'), followed by the necessary attributes to describe the request. These attributes are grouped using a '\&' symbol, and do not require a specific order. Table \ref{table:api_symbols} specifies the possible request attributes, together with the keyword necessary to identify it, and the details and data type of each.

\begin{table}[]
  \centering
  \caption{This table specified the keyword/value pairs, which must be specified as 
  to uniquely identify a resource.}
  \setlength{\tabcolsep}{3mm}
  \label{table:api_symbols}
  \begin{tabular}{llll}
  \hline
  \\[-0.75em]
  Name         & Symbol     & Keyword                                                       & Details                                                                                                                                                                                   \\
  \\[-0.75em]
  \hline
  \\[-0.75em]
  start city   & $v_0$       & flyFrom                                                      & Accepts city name or ICAO code.                                                                                                                                                    \\
  \\[-0.75em]
  return city  & $v_{n+1}$ & returnTo                                                     & Accepts city name or ICAO code.                                                                                                                                                    \\
  \\[-0.75em]
  destinations & $V$          & cities                                                       & \begin{tabular}[c]{@{}l@{}}Defines the cities to be visited. \\ Accepts multiple cities, separated by comma.\\ Accepts city name or ICAO code.\end{tabular}       \\
    \\[-0.75em]
  durations    & $D$          & duration                                                     & \begin{tabular}[c]{@{}l@{}}Defines the length of stay (in days) in each city.\\ Accepts positive integers.\\ Must be the same length as the number of cities.\end{tabular} \\
    \\[-0.75em]
  start date   & $T_0$       & \begin{tabular}[c]{@{}l@{}}minDate \\ maxDate \end{tabular} & \begin{tabular}[c]{@{}l@{}}minDate is the earliest start date ($T_{0i}$),\\ maxDate is the latest start date ($T_{0f}$). \\ Accepts date format dd/mm/yyyy.\end{tabular}  \\
    \\[-0.75em]
    \hline
  \end{tabular}
  \end{table}
  