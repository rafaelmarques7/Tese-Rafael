
\textit{Node.js} \cite{nodejs} is an open-source and cross-platform JavScript run-time environment,
which executes JavaScript code on the server-side \cite{node_article}. Node enables the development of fast and scalable web servers using Javascript only,
by utilizing an asynchronous event driven architecture.
Thus, Node.js enables the unification of web development around a single 
programming language, which may be used both on the client and the server side.
Node.js includes the possibility to use third-party packages, or modules, which can be used for the management 
of networking, file system I/O, data streams, cryptography and much more. These modules are managed and acessible 
through the Node Package Manager, \textit{npm}.

Upon the creation of the first browsers, JavaScript was utilized as a scripting language, that was used to modify, at run time, the Document Object Model (DOM)
of a webpage, enabling the creation of the first dynamic webpages. Today, due to Node.js, this scripting language 
is not restricted to the browser, and can be used in the server to create dynamic web pages, even before the page is served to the client \cite{node_book}.

Node.js was created by Ryan Dahl in 2009, and upon its presentation in the European Javascript Conference,
it utilizes an asynchronous event loop, a low-level Input/Output API, and Google's V8 Javascript engine.
This last technology, the Google's V8 engine, is a fundatamental part of the Node.js stack, because it allows the compilation 
of javascript source code into native machine code, instead of interpreting in real time, as occurs in the browser.

In the development of this work, Node.js is utilized as a way of creating a webserver and serving the User Interface.
Upon a client connection request, the node webserver responds with a javascript bundle file, which containts all the necessary information
for the browser to render, manage and update the UI upon interaction with the user. 