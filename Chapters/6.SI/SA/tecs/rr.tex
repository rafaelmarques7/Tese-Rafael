\textit{React} \cite{react} is a JavaScript library for building and rendering User Interfaces.
React is based on some core principles which dictate the architecture of every UI built using this library,
which include the concepts of \textit{components}, \textit{state}, \textit{JSX} and \textit{Virtual-DOM} \cite{react_redux_article}.

Any User Interface built using React is the result of the \textit{render} method invoked by each of its \textit{components}.
The content that is actually presented by a component is the result of its source code, written in \textit{JSX},
which is a combination of JavaScript and HTML to, dynamically, produce valid HTML elements.
Usually, each of these elements depend on the \textit{state} of the application, at each moment.
Thus, every time the application state changes, the User Interface is updated.
Since re-rendering the entire DOM is very expensive, React uses the so called Virtual-DOM
to re-render only those components which are necessary.
At every state change, the DOM is compared to the virtual-DOM,
and only a branch of the DOM is updated.

As applications grow bigger, it becomes increasingly difficult to manage the state of each component,
as there are natural dependencies amongst components. Currently, the natural response to bypass 
this difficulty is to use a JavaScript library called \textit{Redux} \cite{redux}, even though
there are other solutions developed for this effect.
Redux is a state container, which is often said to be predictable, because it forces 
the usage of immutable data types. This means that every state change is easily identifiable,
which is very convenient for React, due to its Virtual-DOM nature.

Furthermore, Redux is said to be a predictable state container because 
every state change requires the dispatching of a particular \textit{action}, which triggers a function 
that manipulates the state, called \textit{reducer}. Thus, Redux applications
may implement a "history" of the state, because there is information available about every action 
that were executed to reach the current state. 

During the development of this application, React and Redux were used together 
to build the client side application. This means that React is responsible
for rendering the User Interface, while Redux manages the state,
updating it on user actions, and interacts with the client side application
using simple HTTP protocol.

