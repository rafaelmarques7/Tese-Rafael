Django \cite{django} is a python Web framework, designed for rapid development and a clean and pragmatic design. 
It is an oper-source and high-level framework, with dedicated modules for security and performance,
and for the execution of repetitive
task, such as logging, request management, database access, cookies and session management \cite{django_docs}.
Django is constructed with a Model View Controller architecture in mind, or following
the Django nomenclature, a Model View Template (MVT) architecture.

Following the MTV design \cite{learn_django}, upon receiving a request, the webserver first interprets the request based on the 
url, using regular expressions. Upon match, the request is passed to a specific \textit{View}.
A view is the heart of the application, specifying the functions to be executed, and 
returning a response to the request. If the application is database driven,
as most Django applications are, each view may dispatch some \textit{model} construction process,
which queries the database and constructs the necessary data. Finally,
a \textit{template} is populated with the data model, and a response may be returned.

During the development of this application, the server side application is built using Django.
Despite this, most of the functionality that Django presents is not utilized, because it is not completly necessary.
Since the developed application does not directly utilize a Database, the \textit{model} 
part of the MTV architecture may be skipped. Furthermore, the Django's template 
engine may also be completly ignored, because the response to a request is a simple 
python dictionary/JSON object. Thus, the client side application consists in a simple API which interprets requests
according to the specified url, invoking the necessary functions,
and produces a response to each request. 