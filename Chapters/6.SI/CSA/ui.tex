The User Interface consists in a single page application, divided into three main views: the \textit{Request}, \textit{Response} and \textit{Map view}, as proposed in section \ref{sec:csa_design}. Since the application is developed using React, every view is managed by a container, which reads the state from the store, calls the rendering of presentational components, and may dispatch actions on user input or other events. 

The User Interface is designed to be mobile friendly, by being responsive to the  device size. This is achieved using the Bootstrap grid system, a web application design paradigm in which the user screen is divided into 12 columns, and each element of the user interface may specify a variable number of columns, depending on the screen size.

Figure \ref{fig:desktop_app} and \ref{fig:mobile_app} illustrate two possible views of the developed application. The first image illustrates the application in a desktop device and the second in a mobile. It is worth noting that these two screenshots correspond to the same application, and that the design differences between both images are a result of the responsiveness of the application. The responsive design is responsible for resizing the application elements according to the device size and also includes toggles in the \textit{Request} and \textit{Response} views, as to generate more space to the map view. 


\begin{figure}
\centering
\begin{minipage}{.7\textwidth}
  \centering
  \includegraphics[width=\linewidth]{./imgs/bfly_desktop.jpg}
  \captionof{figure}{Application rendered on desktop.}
 % \caption{Screenshot of the developed application rendered on a desktop device.}
  \label{fig:desktop_app}
\end{minipage}%
\begin{minipage}{.3\textwidth}
  \centering
  \includegraphics[width=0.75\linewidth]{./imgs/bfly_mobile.jpg}
  \captionof{figure}{Application rended on mobile device.}
  %\caption{Screenshot of the developed application rendered on a mobile device.}
  \label{fig:mobile_app}
\end{minipage}
\end{figure}

