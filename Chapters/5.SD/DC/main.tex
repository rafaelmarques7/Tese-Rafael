\section{Design Choices}
This section wishes to clarify some of the design choices considered, and the reasons that led to  particular decisions.



\subsection{Problem Formulation}
The formulation of the problem relies heavily on the goal of the overall system.
The start point to this work was the development of an flight search application,
which could solve the multicity trip routing problem,
aswell as any single and round flights,
presenting the best set of flights, that is, those which would minimize 
the total cost of the trip.

To develop rich user experience, that is, a search which is simple and declarative,
and covers a broad number of situations, it is necessary to think like a user.
No matter what trip the user is searching for, there are always some questions which are the same,
and can be grouped into 4 classes: the where, when,  what, and  how.
That is, \textit{where} do we start and finish, 
\textit{when} do we start, \textit{what} cities will are to be visited,
while the \textit{how} may represint a variety of things 
which further characterize the trip, as the duration, possible time windows and/or routing constraints.

In most cases, users have a specific start and returning point, which might or not be the same.
In its turn, the schedule for the trip is usually much more broader.
That is, instead of having a single start date, in some cases user consider a time span of a few days, weeks, or even months. 
Thus, it was decided that instead of forcing a single start date, 
a time period would be considered. 
The same type of reasoning can be applied to the duration of each city.
While some users may have a well defined number of days, say 3,
others might be more flexible and consider staying in the city for 3 to 5 days for example.
This flexibility upon the duration was considered during the development of the formulation,
but not introduced in the overall system, due the extra complexity it would introduce,
and also because it would be difficult to effectively test such a system.
The same can be said about the time windows associated to each city.
While it makes sense to have a broader definition which contemplates
this possibility, it is more convenient to have a simpler but efficient formulation,
which covers only the most important and global search criteria.




\subsection{System Goal}
While the goal of the system was always the development of a flight search engine which would 
always return the cheapest set of flights, upon having a complete system it was noticable 
that in some cases, these set of cheapest flights could be dismissed by the user,
for having some other bad attributes. That is, a flight could be very cheap, but at the expense 
of having a very high flight duration.

To tackle this problem, we propose multiple objective functions which cover different goals.
On one hand, we can define an objective function for the minimization of the total cost,
and upon constructing the TSP problem, we consider the flight cost as the cost matrix entries.
On the other hand, a similar approach can be done to minimize the total flight duration.
Intead of constructing a cost matrix with the flight cost, this is done with the flight duration.
Finally, the same approach is considered for the minimization of the \textit{entropy} of the trip.
We borrow this concept, entropy, to describe the weighted summ of the flight cost and duration.
Selecting the set of flights which have the lowest entropy leads to the increase 
in the \textit{efficiency} of the proposed solution.

Having defined three objective functions, any user request will be responded to with a list 
of three different set of flights, which correspond to the cheapest, fastests, and most efficient solutions found.



\subsection{Data Collection and Storage}
As noted several times, solving a user defined request requires the access to real flight data. 
During the development of this work, two different forms of doing this where tested:
API access and web scraping.
Having access to a flight API is the standart choice, it is fast, customazible and efficient.
However, the number of available free API's is very low,
and overall, the results presented by each may vary considerably.
More than this, the problem is that some API's limit the content they offer,
not including, for example, low cost airlines. This problem was faced specifically with Google Flights Api.
To tackle this problem, and to actually find the cheapest flights,
an alternative was considered, web scraping.

Web scraping, or screen scraping, is the act of visiting a page and copying its content.
This enables the parsing of the page, and the exctraction of usefull information,
which can than be stored into a database, or other form of storage, in order to be accessed.
Thus, web scraping can be used to collect information about the necessary flights,
however, as it was later verified, it is a very complex and specific process.
Furthermore, it is also considerable slow, because most websites use javascript
to produce their information, and this usually requires some access to their database,
which is in general very slow.

Considering the advantages and disadvantages of each system,
it was decided that the developed system would use public API's to access flight data,
because it is much faster and more efficient.

User requests are self contained, that is, the data necessary to process them is gathered,
used, and than discarded. It would be possible to store each collected flight into a database, for later consultation.
However, this data would have an extremely short \textit{expiration date}, or,
in other words, the cost of a particular flight might vary at any moment.
Thus, it can not be proved that the previously stored data is still accurate and up to date.
Because of this, it was opted to not implement a database.
This choice might be adapted at any point, and should be considered if the system actually scales 
to a point in which there are multiple equal queries each day.



\subsection{Optimization System}
Regarding the optimization system, the goal was always the development of a time efficient algorithm.
Producing a complete multi city search is a complex problem, which can take a considerable time.
However, waiting for a considerable time is not an option for most users.
Thus, it is important to produce a solution as soon as possible, and try to improve the solution 
in the subsequent moments. This lead to the system proposed in figure \ref{fig:optimization_system}.



\subsection{User Interface}
Upon the construction of the client side application, which handles the logic of collecting 
user requests, aswell as the rendering of the user interface,
it was important to choose a design which would fit both mobile and desktop users.
There are several market study reports which show that the mobile market segment 
is, and will continue to, experience a growth in its size, textbf{cite}.
Thus, the design of the user interface had to be prepared for the multitude of user devices,
or in other words, it had to be \textit{responsive}, textbf{cite}.





