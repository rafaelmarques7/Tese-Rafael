Upon receiving a user defined request, the set of nodes, as well as the start time and durations, are well defined. On the other hand, the set of arcs which connects these nodes are not. For example, a user request may correspond to a single flight from $A$ to $B$, at time $t$, and, upon receiving this request, there is no available information regarding the possible flights (arcs) between these two cities. In fact, that is exactly what the user is searching for. Thus, the goal of the Data Management System is to collect the necessary information to construct the list of arcs associated to a user request.

In order to access real flight data, it is possible to use a publicly available flight data API. This is the simplest, fastest and most efficient way to request flight data. There are several possible choices upon selecting a flight API. In summary, the available API's may be classified as \textit{free}, \textit{limited} and \textit{enterprise}. A free API is one which does not charge or limit the available queries. On the other hand, a limited API is one which sets an upper bound on the number of daily available queries, and charges a fee after this limit. In its turn, an enterprise API is available  only for commercial solutions, and it may not be used in a research context. During the development of this work, only free API's were considered. 

Communication with a flight API utilizes a simple HTTP protocol, and every request is identified according to a Uniform Resource Identifier, whose syntax is defined by the API provider. A response to a request usually consists in a data tree, using a structured data format, as JSON or XML. Every response includes a list of possible flights, and each flight has a vast number of attributes, as the cost, flight duration, departure time, and so on.

Although there are several publicly available flight data API's, the information provided by each of those might be considerable different. Multiple flight search applications were compared, and the results were presented in tables \ref{tab:simple_flights_analysis} and \ref{tab:multi_city_analysis}. The analysis of these results shows that the flight data presented by each of these API's varies considerably. For example, the cost of a single flight may be up to 44\% higher, according to the API flight data provider. 

One of the main goals of the proposed web application is to find the best set of flights for a given query, according to some objective function and, in particular, the minimization of the total flight cost. Given that there are considerable differences among flight API's, ideally, the proposed web application should query multiple API's.

The role of the Data Management System is of crucial importance for the development of a high quality flight search web application. This is because every user request seeks to find a given set of flights to a satisfy a particular itinerary. Thus, it is of extreme importance to have the most up-to-date flight data for each of the possible flights.  

