\textit{Node.js} \cite{nodejs} is an open-source and cross-platform JavaScript runtime environment, which executes JavaScript code on the server \cite{node_article}. Node enables the development of fast and scalable web servers using JavaScript only, by implementing an asynchronous event loop, a low-level input/output API, and Google's V8 JavaScript engine. This last technology is a fundamental part of the Node.js stack, since it allows the compilation of JavaScript source code into native machine code. 

Upon the creation of the first web browsers, JavaScript was utilized as a scripting language used to modify, in run time, the content of a web page. This enabled the creation of the first dynamic web pages. Today, due to Node.js, this scripting language is not restricted to the browser, and can be used in the server to create dynamic web pages, even before the page is served to the client \cite{node_book}. 

Thus, Node.js enables the unification of the web development environments around a single programming language, which may be used both on the client and the server. It also includes a package manager (npm) that allows the execution of third-party software, that can be used, for example, in the management of databases, networking, file system I/O and data streams.

In the development of this work, Node.js is used to create a web server for the Client Side Application. Upon receiving a client request, the web server responds with a JavaScript and \ac{HTML} bundle file. This bundle is the result of the compilation of the dynamic React application, illustrated in figure \ref{fig:react_redux_app} in the following section. This bundle contains the necessary information for the browser to render, manage and update the User Interface upon interaction with the user and third-party APIs.