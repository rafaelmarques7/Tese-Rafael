Given a list of cities and the distances between them, the Traveling Salesman Problem is the combinatorial optimization problem of finding a minimum length route which connects every city. With this original definition, proposed by \cite{cite_1}, the focus of the TSP is to perform optimization on routing problems, as the school bus problem, studied by Merrill Flood in 1942, \cite{cite_2}, minimizing the total distance of a tour. Some variations of the original formulation allow the adaptation of the problem to suit different optimization goals, \cite{cite_3}. For example, instead of distance, the focus may be the minimization of the total cost, travel time, or some other attribute associated to the problem under consideration. It is also possible to search for a route which minimizes two, or more, objective functions at once, \cite{cite_4}.  In some routing problems, the tour under consideration must satisfy some constraints, \cite{cite_8}. Most often, these constraints refer to scheduling conflicts which must be satisfied, \cite{cite_9}. A practical application of this is the resolution of routing problems with time windows, \cite{cite_10}.

When states in the field of Graph theory, the TSP is the problem of finding a minimum cost Hamiltonian cycle over a complete, undirected, weighted graph, \cite{cite_5}. The problem of finding a minimum cost Hamiltonian cycle was shown to be NP-complete. This implies the NP-hardness of the TSP \cite{cite_6}. Furthermore, in several graph problems, considering a symmetric cost between two points is not suitable. This is known as the asymmetric TSP, and it considers a directed graph instead \cite{cite_7}.

In the flight industry, the Traveling Salesman has vast applications. As an example, it was applied to aircraft scheduling in the terminal area, enabling the increase in the airports capacity \cite{cite_14}. More recently, the TSP, and its time dependent variation, have been focus of attention in fields related to Unmanned Aerial Vehicle routing (\cite{cite_16}, \cite{cite_17}). There is also an online website which introduces the Air Traveling Salesman, whose goal is to find the layover airports when no direct route is available \cite{cite_15}.

In some cases, the classic formulation of the Traveling Salesman does not adequately describe the characteristics of the problem under consideration. To overcome this, different problem formulations are considered. An example of this is the time dependent TSP \cite{cite_12}. In this formulation, the cost of each arc is not constant, but varies as a function of time. In general, this problem is harder to solve than the classic TSP \cite{cite_21}. There are several other combinatorial optimization problems which benefit from considering a time dependent approach \cite{cite_20}. Vehicle routing is a field which particularly focus on this problem, due to the characteristics of street traffic \cite{cite_19}. In fact, the Traveling Salesman Problem can be regarded as a special case of the Vehicle Routing Problem, in which the fleet is composed by only one vehicle, the salesman, \cite{cite_19}. Because of this, works related to the Vehicle Routing Problem may also be relevant for the resolution of the Traveling Salesman Problem.

During the day to day activities, many people are faced with similar routing problems every day. Consider the problem of walking or driving from point A to point B. This is a graph problem, in which the arcs are the streets, the nodes the streets intersections \cite{cite_11}. In its turn, the weights refer to the distance or travel time, which in its turn may be affected by other parameter, as traffic \cite{cite_12}. If the person is familiar with the graph, they are capable of finding a good route mentally, in a very fast manner \cite{cite_13}. If the undertaken route is to visit a set of points exactly once, before returning to the original starting point, this is known as the Traveling Salesman Problem. 

This section is structured as follows. First, we formally define the classic Traveling Salesman Problem, in section \ref{sec:classic_tsp}, and present some of its most common variations. This is followed by a more in-depth overview of the time-dependent TSP, in section \ref{sec:time-dependent-tsp}, as this problem is particularly relevant for the work under development. For the same reason, the following subsections \ref{sec:time-windows-tsp} and \ref{sec:multi-objective-tsp} present the TSP with time-windows and with multiple objectives, respectively.