\subsubsection{Problem definition}
\label{sec:classic_tsp}

As it was previously referred, the Traveling Salesman may be defined both as combinatorial optimization and as a graph problem. In either case, the TSP is defined by a graph $G = (N, A)$, where $N$ is the set of nodes, and $A$ is the set of arcs connecting those nodes. The set of nodes is of size $n = |N|$, while the size of the set of arcs is $m = |N|^2$. Each arc, $a_{ij}, i, j \in N, i \ne j$ has an associated weight, $c_{ij}$, which represents, for example, the distance between cities. The set of arcs is fully connected, that is, each node is capable of directly reaching any other node, without visiting a third node. When two nodes can not be connected by an arc, the cost of that node is considered as very high. In the classical TSP formulation, the graph is undirected, which implies symmetry in the costs of the arcs, that is: $c_{ij} = c_{ji} \forall i, j \in N$. Because of the characteristics of this TSP formulation, the graph is said to be connected, weighted and undirected.

The objective of the TSP is to find the minimum cost Hamiltonian cycle, that is a path which visits each node exactly once, and returns to the initial node, closing the path. A generic solution to the TSP is any permutation $\sigma$ over the set of nodes, $N$. The permutation $\sigma$ is also a set, where $\sigma_{i}$, $i$ $\in$ $len(\sigma)$, represents the node in the $i$'th index of the cycle. The cost of a cycle is given by the sum of the weights of each arc by which it is composed, that is $C(\sigma) = \sum_{i=1}^{n} c_{\sigma_{i} \sigma_{i+1}}$, where $\sigma_{n+1} = \sigma_{1}$

The TSP may solve different types of problems by optimizing different parameters. In the classic formulation, the weight between an arc connecting two nodes represent the distance between two cities. However, the weight of an arc can represent different things, particularly, travel time or travel cost. Although changing the parameter may lead the TSP formulation intact, in some cases, it changes the problem. This occurs, for example, when considering that the costs are time-dependent and this will be approached in section $\ref{sec:time-dependent-tsp}$.