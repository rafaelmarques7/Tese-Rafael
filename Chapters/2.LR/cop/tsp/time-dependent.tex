\subsubsection{Time Dependent TSP}
\label{sec:time-dependent-tsp}

The time-dependent traveling salesman problem (TDTSP) is a generalization of the TSP, where arc costs depend on their position in the tour \cite{time_dependent_tsp, tdtsp_single_machine}. This section first introduces the TDTSP as a graph problem, followed by its definition as a sequencing problem. 

\textbf{TDTSP as a graph problem}

Let $N = {1, 2, ..., n}$ and let $N_{0} = N \cup \{0\}$. The TDTSP on a complete graph $K(N_{0})$ can be modeled as an optimization problem over a layered graph $(V, A)$. $V$ is the set composed by the source node $0$, the termination node $T$, and intermediate nodes $(i, t)$ for $i, t \in N$. In this representation of the intermediate nodes, the first index of $(i, t)$ identifies the vertex $i$ of the graph $K(N)$, and the second index represents the position of vertex $i$ in a path between nodes $0$ and $T$. In its turn, $A$ is the set of arcs connecting each node. This set is composed of initiation, intermediate, and termination arcs. For $i \in N$, $(0, i, 0)$ denotes an initiation arc from node 0 to node $(i, 1)$, and $(i, T, n)$ denotes a termination arc from node $(i, n)$ to node $T$. Given $i, j \in N$ such that $i \neq j$, and $1 \leq t \leq n-1$, $(i,j,t)$ denotes an intermediate arc from node $(i,n)$ to node $(j, t+1)$. The third index of an arc represents its layer, that is, the position in which it occurs in the path, or in other words, the time of the arc traversal, if we consider 1 time unit for each of these traversals.

When working on the TDTSP, it is often convenient to define $G(n)$ as a subgraph of $(V, A)$, induced by $V\backslash\{0, T\}$. \todo{$V\backslash\{0, T\}$ is the set V without the source and termination node}This way, $G(n)$ has $n^2$ nodes ${(i, t): i, t \in N}$ and all the $n(n-1)^2$ intermediate arcs of $A$. A path with $n$ vertices in $G(n)$ is of the form ${(v_{t}. t): v_{t} \in N, 1 \leq t \leq n}$. Since consecutive nodes are in consecutive layers, the path can be described by an ordered array $(v_{t}: t \in N)$. This path can be extended to a $0-T$ path of $(V,A)$ by appending node $0$ and $T$ to the beginning and end of the tour, respectively.

The classical TSP and its time-dependent variation share the same objective, which is to find the minimum cost Hamiltonian cycle over graph $(V, A)$. Another property they share is the possibility of working over a symmetric or asymmetric problem.


\textbf{TDTSP as one-machine sequencing problem}

In operation scheduling, the time dependent TSP can also be stated as a one-machine sequencing problem \cite{tdtsp_single_machine}. Consider a set of $n$ jobs, $J_{1}$, ..., $J_{n}$, to be executed on a single machine. Each job has a setup cost $C_{i,j}^{t}$, occurring when job $J_{i}$, processed in the $t$-th time unit, is followed by job $J_{j}$, processed in the $(t+1)$-th time unit. Consider that each job completion takes exactly one time unit. The machine is in some given initial state, denoted by 0, before the job processing begins. As happens with the classical TSP, the machine has to be returned to its original state, after the job processing ends. The problem is to find a sequence, $J_{w(1)}$, ..., $J_{w(n)}$, that minimized the total set-up cost C(w), defined by:

\begin{equation}
  C(w) = C_{0, w(1)}^{0} + \sum_{i=1}^{n-1} C_{w(i), w(i+1)}^{i} +  C_{w(n), 0}^{n}
\end{equation}

It is important to note some important characteristics of this formulation. First, problems with unspecified initial/final state can be formulated in the same way using 0 as the initiation/termination cost, that is $C_{0, w(1)}^{0} = 0$ and $C_{0, w(1)}^{n} = 0$. Secondly, the overwriting of the above formulation reduces the defined problem into the classical TSP and the classical Assignment Problem \cite{assignment_problem}. The first case is achieved by considering that setup costs are not time-dependent, that is, $C_{i, j}^{t} = C_{i, j}$. The latter is accomplished by considering that setup costs are not dependent on the second/first job, that is, $C_{i, j}^{t} = C_{i}^{t}$.