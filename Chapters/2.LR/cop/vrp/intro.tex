The Vehicle Routing Problem is the problem of finding the optimal set of routes for a fleet of vehicles,
to serve a given set of customers.
The VRP is believed to be introduced by Dantzig, in 1959, in a work with the name of \textit{The Truck Dispatching Problem},
in which it is considered a generalization of the Traveling Salesman, \cite{truck_problem_dantzig}. 
It was latter shown that, being a generalization of the TSP, its computational complexity is also NP-hard, \cite{VRP_complexity}. 

Being an NP-hard problem, the focus of the research usually revolves around heuristic algorithms, although there are some procedures
which are known to produce optimal solutions, \cite{VRP_exact_heuristic}, \cite{TDVRP_exact}. As refered by Donati in \cite{MACS_VRPTW}, citing the work of Blum, \cite{blum_complexity},
even when an exact procedure is available, it usually requires large computational time, which is not viable in the time-scale of hours, as required by this industry. 

Malandraki, \cite{TDVRP_92}, as early as 1992, stated that the assumption of constant and deterministically known costs,
is an approximation of the actual conditions of routing problems, and thus, a time-dependent formulation of the problem should be considered.
In 1999, Gambardella and colleagues proposed a multi ant colony system for solving the vehicle routing problems using a meta-heuristic approach, \cite{MACS_VRPTW}. 
Years later, Gambardella expanded this research to include time-dependent variations, \cite{TDVRP_multi_objective_aco}, as proposed by \cite{TDVRP_92}. 
There are several other works, which propose meta-heuristic solutions to solve the time-dependent VRP, \cite{VRP_meta}, including the use of simmulated annealing, \cite{tdvrp_costs},
and genetic algorithms, \cite{TDVRP_GA}. 

The rest of this section is structured as follows. The next section presents a formal definition of the Vehicle Routing Problem and its time-dependent variation,
as well as the most common objectives of the resolution of this problem. Since the TSP occurs only as a generalization of the non-capacitated vehicle routing problem,
the study of the capacited vehicle routing is out of the scope of this work.
