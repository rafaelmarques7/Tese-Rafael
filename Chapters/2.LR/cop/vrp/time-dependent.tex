\subsubsection{Time-dependent VRP}
\label{sec:time-dependent-vrp}


The Vehicle Routing Problem is a very wide class of optimization problems, whose precise problem definition usually depends
on the characteristics of the problem under considerations. Thus, introducing time-dependencies 
on the problem also depend on the specifics of the situation. There are several athors which consider 
time-dependent travel costs, \cite{tdvrp_costs}, and the objective is to minimize the total costs,
while others introduce time-dependent travel times, \cite{tdvrp_1},
and the objective is to minimize the total travel time. There are also those 
who consider that the objective function is a function of both travel time and travel costs,
and at least one of these (travel time, travel cost) is time-dependent, \cite{tdvrptw_1}.
The definition here proposed follows this last time-dependent variation.

The time-dependent VRP is defined, following the work of \cite{tdvrptw_1}, as follows.
Let $G=(V,A)$ be a graph where $A=\{(v_i, v_j): i \neq j \wedge i,j \in V\}$
is the set of arcs, and $V = (v_0, ..., v_{n+1})$ is the set of vertex
Vertices $v_{0}$ and $v_{n+1}$ denote the depot at which the vehicles are based.
It is considered that each vehicle has an uniform capacity of $q_{max}$.
It is also expected that each vertex in $i \in V$ has an associated demand $q_i \geq 0$,
a service time $g_i \geq 0$, and the depot has $q_0 = 0$ and $g_0 = 0$.
The set of vertex $ C = (v_1, ..., v_{n})$ specified the set of $n$ customers.
The arrival time of a vehicle at customer $i$, $i \in C$, is denoted by $a_i$,
and its departure time $b_i$. Each arc $(v_i, v_j)$ has an associated distance $d_{ij} \geq 0$,
and a travel time $t_{ij}(b_i) \geq 0$. Note that the travel time is a function 
of the departure time from costumer $i$.
The set of available vehicles is denoted by $K$. 
Consider that the cost per unit of route \textit{duration} is denoted by $c_{t}$,
and the cost per unit of route \textit{distance} is denoted by $c_{d}$. 

In this formulation, there are two goals for the time-dependent VRP. The first corresponds
to the minimization of the total number of vehicles used. The second corresponds
to the minimization of the total cost, which is a function of both distance and travel time.

There complete definition of the problem follows a mixed integer programming approach,
with a total of 11 constraints. These will not be covered in detail here, as the VRP is not 
the primary object of study of this work, that being the TSP. Thus,
it is important to define in which circustances the TDVRP can be transformed into the TDTSP.
This is possible by considering only one vehicle, with infinite capacity,
and by adapting the objective function according to the problem under consideration.

We conclude this section with the final remark that the above presented definition of the 
time-dependent VRP corresponds to a static version of the time-dependent case.
There is a lot of research around the dynamic case, in which the problem is updated
during the execution of the program. This has major applications in the routing industry,
and it is often refered to as \textit{real-time} Vehicle Routing. 
For more information regarding this problem, we refer to \cite{real_time_vrp} \cite{dynamic_vrp}.



