\subsubsection{Held-Karp Lower Bound}

In some cases, the quality of a heuristic solution can not be directly measured, as no exact solution for the problem under consideration is known. In these cases, it is important to have a way of evaluating its performance. The standard way of doing this is by comparing the heuristic solution with the solution generated by the Held-Karp (HK) lower bound \cite{held_karp_lb}.

The HK lower bound is the solution of the linear programming relaxation of the ILP formulation of the TSP. This solution can be found in polynomial time for moderate instance sizes. However, for a very large problem, solving the relaxed problem directly is not feasible. In these cases, Held and Karp proposes an iterative algorithm in order to approximate the solution. This method involves computing a large number of minimum spanning trees. This iterative version of the algorithm will often keep the solution
within $0.01\%$ of the HK lower bound \cite{heuristics_tsp}.
