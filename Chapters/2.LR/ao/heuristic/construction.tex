\subsubsection{Tour construction}


A construction algorithm is based on the construction of a valid tour.
The construction process stops when a valid tour is found. No improvement
over the formulated tour is attempted.

\paragraph{Nearest neighbour}
The nearest neighbour is a very simple heuristic for the TSP. This algorithm
starts with the selection of a random node. Then, while there are unvisited nodes,
the heuristic always selects the nearest node which was not yet visited.
This proccess is repeated while there are unvisited nodes. Finally, when there
are none, the construction is complete with the return to the first node.

The computational complexity of the nearest neighbour is $\mathcal{O}(n^2)$.
The solutions generated with this heuristic are often within $25\%$ of the
optimal solution.

The pseudocode for the NN algorithm is presented below.

\begin{enumerate}
  \item Select a random city
  \item Select the nearest unvisited node
  \item If there are unvisited nodes, repeat step (2)
  \item Return to first node
\end{enumerate}


\paragraph{Greedy heuristic}
The greedy heuristic is a construction algorithm which creates a valid tour
by repeatedly selecting the arc with the lowest weights, and taking into account
the problems constraints. In particular, the greedy algorithm rejects an arc
which creates a cycle with less than $n$ edges, or which would create a sub tour.

The computational complexity of the greedy heuristic is $\mathcal{O}(n^2log_2(n))$.
The solutions generated by this heuristic are often within the $20\%$ of the
optimal solution.

\begin{enumerate}
  \item Sort all arcs according to its weight
  \item Select the lowest weight arc, if it does not violate any constraint
  \item If the constructed solution is not complete, repeat (2)
\end{enumerate}
