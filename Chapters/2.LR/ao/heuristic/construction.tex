\subsubsection{Tour construction}

A construction algorithm is based on the generation of a valid solution, by using some heuristic function to guide the construction process. These algorithms stop when a valid solution is found, and they do not attempt to improve the solution any further. Two different tour construction heuristics for the TSP will be presented: the nearest neighbour and the greedy heuristics.

The nearest neighbour \cite{local_search_book} is a simple and intuitive heuristic for the TSP. It starts with the selection of a random node. This is followed by the selection of the closest node, belonging to the set of nodes not yet visited. This step is repeated until all nodes have been visited. Finally, the solution construction process is completed by returning to the initial node. The computational complexity of the nearest neighbour is $\mathcal{O}(n^2)$, and the set of solutions generated with this heuristic are often within $25\%$ of the optimal solution \cite{heuristics_tsp}. The pseudocode for the nearest neighbour heuristic is presented below.

\begin{enumerate}
  \item Select a random city
  \item Select the nearest unvisited node
  \item If there are unvisited nodes, repeat step (2)
  \item Return to first node
\end{enumerate}

In its turn, the greedy heuristic \cite{local_search_book} is a construction algorithm which creates a valid solution by repeatedly selecting the arc with the lowest weights, always taking into account the problem constraints. Although this heuristic is similar to the nearest neighbour, there are some differences in the initialization step. The nearest neighbour randomly selects the initial node, while the greedy heuristic is greedy at every step of the algorithm, including the initialization. The computational complexity of the greedy heuristic is $\mathcal{O}(n^2log_2(n))$, and the solutions generated by this heuristic are often within the $20\%$ of the optimal solution \cite{heuristics_tsp}. The pseudocode for the greedy heuristic is presented below. 

\begin{enumerate}
  \item Sort all arcs according to its weight
  \item Select the lowest weight arc, if it does not violate any constraint
  \item If the constructed solution is not complete, repeat (2)
\end{enumerate}
