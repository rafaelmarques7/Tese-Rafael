\subsubsection{Integer Linear Programming}
\label{sec:TDTSP_ILP}

The Traveling Salesman Problem, defined over the graph $G=(V,A)$, may be formulated as an integer linear programming problem \cite{tsp_exact_review}, by associating one binary decision variable $x_{ij}$ to every arc $a_{ij}$. Let $c_{ij}$ represent the weight of the arc $a_{i,j}$. When a decision variable has a value of 1, the corresponding arc belongs to the solution. The ILP formulation for the TSP is as follows:

\begin{align}
  \min  &\hspace{0.5em} \sum_{i \ne j}c_{ij}x_{ij} \label{eq:ILP_objective} \\
  \mathrm{s.t.}   &\hspace{0.5em} \sum_{j} x_{ij}, \hspace{3mm} i = 1, \ldots, n, \label{eq:ILP_constraint_1}\\
  & \hspace{0.5em} \sum_{i} x_{ij} = 1, \hspace{3mm} j = 1, \ldots, n, \label{eq:ILP_constraint_2}\\
  & \hspace{0.5em} \sum_{i,j \in S} x_{ij} \leq |S|-1, \hspace{3mm} S \subset V, 2 \leq |S| \leq n-2 \label{eq:ILP_constraint_3}\\
  & \hspace{0.5em} x_{ij} \in \{0, 1\}, \hspace{3mm} i, j = 1, \ldots, n, i \ne j. \label{eq:ILP_constraint_4}
\end{align}

The objective function of the TSP is described by equation $\ref{eq:ILP_objective}$. Equations $\ref{eq:ILP_constraint_1}$ and $\ref{eq:ILP_constraint_2}$ represent the imposed constrains over the variables. In particular, they state that a tour must enter and leave, respectively, each node exactly once. However, this does not completely define the characteristics of a Hamiltonian cycle. To eliminate the possibility of subtours, it is necessary to introduce the sub-tour elimination constraint, expressed in equation $\ref{eq:ILP_constraint_3}$, . The final constraint, defined in equation \ref{eq:ILP_constraint_4}, forces the decision variables to binary values.


% _______________________________________________________________________________
% _______________________________________________________________________________

In its turn, the time-dependent Traveling Salesman Problem variant may be formulated as an integer linear programming problem by associating one binary decision variable $x_{ijt}$ to every arc $a_{ijt}$. Let $c_{ijt}$ represent the weight of the arc $a_{i,jt}$. A decision variable $x_{ijt}$ takes a value of 1 when the arc $a_{ijt}$, which represents the transition from node $i$ to node $j$ at time $t$, belongs to the solution. Hence, by following a similar ILP formulation for the TDTSP: \todo{Rever esta frase}

\begin{align}
  \min  &\hspace{0.5em} \sum_{i}\sum_{j}\sum_{t} c_{ijt}x_{ijt} \label{eq:objective_TDTSP_ILP} \\
  \mathrm{s.t.}   &\hspace{0.5em} \sum_{j}\sum_{t}x_{ijt} = 1 , \hspace{3mm} i = 1, \ldots, n, \label{eq:TDTSP_ILP_constraint_1}\\
  & \hspace{0.5em} \sum_{i}\sum_{t}x_{ijt} = 1, \hspace{3mm} j = 1, \ldots, n, \label{eq:TDTSP_ILP_constraint_2}\\
    & \hspace{0.5em} \sum_{i}\sum_{j}x_{ijt} = 1, \hspace{3mm} t = 1, \ldots, n, \label{eq:TDTSP_ILP_constraint_3}\\
  & \hspace{0.5em} \sum_{j = 1}^{n}\sum_{t=2}^{n}tx_{ijt} - \sum_{j = 1}^{n}\sum_{t=1}^{n}tx_{ijt}  = 1, \hspace{3mm} \quad i = 1, ... ,n \label{eq:TDTSP_ILP_constraint_4}\\
  & \hspace{0.5em} x_{ijt} \in \{0, 1\}, \hspace{3mm} i, j, t = 1, \ldots, n, i \ne j. \label{eq:TDTSP_ILP_constraint_5}
\end{align}

The objective function is presented in equation $\ref{eq:objective_TDTSP_ILP}$. Equations $\ref{eq:TDTSP_ILP_constraint_1}$, $\ref{eq:TDTSP_ILP_constraint_2}$ and $\ref{eq:TDTSP_ILP_constraint_3}$ represent the imposed constraints over the decision variable. Particularly, they state that each city must be entered exactly once, left exactly once, and visited in exactly one time period, respectively. As occurs with the classical TSP, the ILP formulation needs to formulate a constraint to eliminate the  formation of subtours. This is presented in equation $\ref{eq:TDTSP_ILP_constraint_4}$. Finally, equation $\ref{eq:TDTSP_ILP_constraint_5}$ guarantees that the decision variable takes binary values.
