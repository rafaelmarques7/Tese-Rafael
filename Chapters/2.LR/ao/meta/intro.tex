Meta-Heuristic algorithms are heuristic algorithms which, unlike the classical heuristic,
can be applied to a variety of optimization problems. Meta-Heuristic are designed to be applied 
to combinatorial optimization problem, and not to a specific problem of this class. 
Meta-Heuristic rose in importante during the 1990's, and have become one of the most important class 
of algorithms in computer science.

More formally, a meta-heuristic is an iterative generation process, which guides a heuristic 
by combining intelligently different concepts, for exploring and exploiting the search space,
using learning strategies to structure information, as to efficiently find optimal or near-optimal solutions, \cite{metaheuristics_overview}.

This subsection will introduce a few of the most relevant meta-heuristics in the resolution of the 
Traveling Salesman Problem, particularly, the Ant Colony Optimization (ACO) and the Simmulated Annealing procedures (SA).
There is a variety of meta-heuristics which are not discussed here, but which have also been succesfully applied 
to the TSP. Examples of these meta-heuristics are the Tabu-Search, Evolutionary Algorithms (EA), in particular the Genetic Algorithm (GA),
and many other Swarm Intelligence algorithms, from which the Ant Colony Optimization is the oldest and most widely used.
