Unlike classic heuristics, metaheuristic algorithms are designed to be applied to any combinatorial optimization problem, and not to a specific problem of this class. Meta-Heuristic gain importance during the 1990's, and have become one of the most important class of algorithms in computer science.

More formally, a meta-heuristic is an iterative generation process, which guides an underlying heuristic by combining intelligently different concepts, for exploring and exploiting the search space, using learning strategies to structure information, as to efficiently find optimal or near-optimal solutions \cite{metaheuristics_overview}.

This subsection will introduce a few of the most relevant meta-heuristics in the resolution of the Traveling Salesman Problem. In particular, it will focus on the Ant Colony Optimization (ACO) and the Simulated Annealing (SA), presented in subsections \ref{sec:aco_lr} and \ref{sec:sa_lr}, respectively. There is a variety of meta-heuristics which are not discussed here, but which have also been successfully applied to the TSP. Examples of these meta-heuristics are the Tabu-Search, Evolutionary Algorithms (EA), in particular the Genetic Algorithm (GA), and many other Swarm Intelligence algorithms, from which the Ant Colony Optimization is the oldest and most widely used.
