% #############################################################################
% Abstract Text
% !TEX root = ../main.tex
% #############################################################################
% use \noindent in firts paragraph

\noindent 

%1) Introduction - find the best route, schedule and set of flights for a multi-city trip.
%2) Overview of related problems - gives an idea on what strategies to follow
%3) Formal Definition of the problem and proposed methods to solve it
%4) Designing and implementation a web application for it
%5) Experimental results - optimization system performance, utility of the proposed problem, and comparison to current state of the art.

%- assim, a estrutura seria:
% a) descrição do "problema" (unconstrained flying search), e da 
% dificuldade em resolvê-lo
% b) semelhanças com o TSP
% c) só aqui dizes que as aplicações online, apesar de oferecerem serviços 
% básicos de procura, não oferecem (na generalidade) este tipo de serviço, 
% dada a grande complexidade do problema em questão.
% d) a partir daqui dizes o que é que fizeste (enumeras, não é necessário 
% explicar):
% - formulação do problema FTP
% - proposta de metodologia para o resolver (base on...)
% - implementação de um protótipo sob a forma de uma aplicação web
% e) parágrafo acerca dos resultados. Não deves dizer que os resultados 
% são "positivos"!!! Isto não diz nada!!! Em vez disso, deves apresentar 
% resultados quantitativos (com números). Sugiro que olhes para as várias 
% secções do capítulo dos resultados e que incluas aqui uma frase com o 
% resultado (número) mais significativo para cada vertente dessa avaliação.
% f) podes terminar o abstract com uma frase a referir o impacto do teu 
% trabalho na perspectiva do utilizador


The present work formalizes and addresses the Flying Tourist Problem (FTP), a NP-hard problem that occurs as a generalization of the Traveling Salesman Problem (TSP), and whose goal is to find the best schedule, route, and set of flights, for any given unconstrained multi-city flight request. In fact, despite the current existence of numerous flight search applications, most of them lack the ability to  properly address unconstrained multi-city flight requests, since this problem is generally not tractable. In accordance, the main goal of this research is to develop a methodology that allows an efficient resolution 
of this rather demanding problem. To accomplish this, different heuristics and meta-heuristic optimization algorithms were considered, including the Ant Colony Optimization and the Simulated Annealing, allowing the identification of solutions in real-time, even for large instances. The developed methods were integrated into a web application prototype, allowing a fast resolution of user-defined requests. In particular, the implemented system was evaluated using different criteria, including the quality of its optimization system; the utility of the devised problem; and its performance compared to other similar systems. The obtained results show that the developed optimization system consistently presents solutions that are as close as 10\% to the optimal value and the considered meta-heuristic optimization strategies present solutions that are up to 35\% cheaper than those developed by simpler heuristics. Furthermore, when comparing the developed system to  the only known (but not-disclosed) alternative, it was shown that the developed application provides the cheapest and the best-recommended solutions, respectively 95\% and 74\% of the times. As a result, upon  the planning of a complex multi-city trip, the developed system showed to allow the user to save a significant amount of time and money.

%Keywords: Flying Tourist Problem, Traveling Salesman Problem, 
%Combinatorial Optimization, Ant Colony Optimization, Simulated 
%Annealing, Web Application.

% The present work addresses the Flying Tourist Problem, a problem that occurs as a generalization of the Traveling Salesman Problem, and whose goal is to find the best schedule, route and set of flights, for any given unconstrained multi-city flight request. 
% Despite the existence of numerous flight search applications, most of them lack the ability to properly address unconstrained multi-city flight requests, since this problem is intractable. 
% Hence, the goal of this work is to develop a tool that allows an efficient resolution of this problem.
% To accomplish this, the developed work uses different heuristics and meta-heuristic optimization algorithms, including the Ant Colony Optimization and the Simulated Annealing. 
% This allows the construction of solutions in real-time, even for large instances. 
% These developed methods are integrated into a web application prototype, enabling the resolution of user defined requests. 
% This system is evaluated using different criteria: the quality of its optimization system; the utility of the devised problem; and its performance compared to other similar systems. 
% The results show that the developed optimization system consistently presents solutions within 10\% optimality, and that the meta-heuristic optimization strategies present solutions up to 35\% cheaper than those developed by simpler heuristics. 
% Furthermore, upon comparing the developed system to the only known alternative, it was shown that the developed application presents the cheapest and the best recommended solutions, respectively 95\% and 74\% of the times.
% Thus, upon the planning of a complex trip, the developed system allows the user to save time and money.

% Keywords: Flying Tourist Problem, Traveling Salesman Problem, Ant Colony Optimization, Simulated Annealing, Combinatorial Optimization, Flight Search.

%The comparison of the developed system to the only other known service that implements a solution to the problem, shows that the developed application presents the cheapest and the best recommended solution, respectively 95\% and 74\% of the times. 
%Furthermore, the developed system presents the cheapest and the best recommended solutions, respectively 95\% and 74\% of the times, when compared to the other known service that implements a solution to the problem.

%This work aims at the resolution of the \textit{unconstrained} multi-city flight 

% The present work addresses the Flying Tourist Problem, whose goal is to find the best schedule, route and set of flights to any given unconstrained multi-city flight request. 
% This problem occurs as a generalization of the Traveling Salesman Problem and its time-dependent variation, and is particularly hard to solve as the problem size increases. 
% Despite the existence of numerous flight search applications, most of them lack the ability to properly address multi-city flight requests. 
% Hence, the goal of this work is to develop an application that allows the resolution of this problem, by finding the best route, schedule and set of flights, for any unconstrained multi-city flight request. 
% To accomplish this, the developed work uses different heuristics and meta-heuristic optimization algorithms, including the Ant Colony Optimization and the Simulated Annealing. Using different optimization strategies, it is possible to construct solutions in real-time, even for large problem instances. Finally, these methods are integrated into a web application prototype, that allows the resolution of user defined requests. The developed system is evaluated using three different criteria: the quality of its optimization system; the utility of the devised problem; and its performance compared to other similar systems. The results show that the developed optimization system consistently presents solutions within 10\% optimality. The results also show that the meta-heuristic optimization strategies present solutions up to 35\% cheaper than those developed by simpler heuristics. The comparison of the developed system to the only other known service that implements a solution to the problem, shows that the developed application presents the cheapest and the best recommended solution, respectively 95\% and 74\% of the times. Thus, upon the planning of a complex trip, the developed system allows the user to save time and money.

% The considered problem belongs to the class of scheduling and routing problems. In particular, it occurs as a generalization of the Traveling Salesman Problem and its time-dependent variation. 
% %Thus, an overview of the methodologies implemented in the resolution of these problems may be useful for the definition of an adequate strategy to the resolution of the considered problem. 

% This work proposes a formal definition for the problem under resolution, and denotes it as the Flying Tourist Problem. In order to solve it, this work uses different heuristics and metaheuristics, including the Ant Colony Optimization and the Simulated Annealing. Using these methods, a solution may be constructed in real-time, even for large problem instances.

% The methods developed for the resolution of this problem are integrated into a web application prototype, making it publicly available.

% The developed system is evaluated using three different criteria: the quality of its optimization system; the utility of the devised problem; and its performance compared to other similar systems. The results are extremely positive, and show that it is possible to
% considerable reduce the price and flight duration, even for small problem instances. 

% Keywords: Flight search application, Routing and scheduling problem, metaheuristics

%reduce the cost in 15\% on a 5 cities trip, and 33\% on a 10 cities trip.

%The performance of the developed optimization methods is evaluated using a set of benchmark tests, while the utility of the proposed system is determined according to the price savings, when compared to the standard way of solving these type of requests. 
%The results presented in this work are extremely positive, showing that is is possible to reduce the cost in 15\% on a 5 cities trip, and 33\% on a 10 cities trip.
%The results presented in this works are extremely positive, and show that, as the number of cities to visit increases, the proposed system is capable of presenting solutions with a lower flight price and duration, when compared to the standart me


%This problem occurs as a generalization of the Traveling Salesman Problem and thus, an overview of this problem and the methodologies used in its resolution may be useful for the definition of an adequate strategy for the resolution of the considered problem.




% How to Write an Abstract for Your Thesis or Dissertation
% What is an Abstract?
% The abstract is an important component of your thesis. Presented at the beginning of the thesis, it is likely the first substantive description of your work read by an external examiner. You should view it as an opportunity to set accurate expectations.
% The abstract is a summary of the whole thesis. It presents all the major elements of your work in a highly condensed form.
% An abstract often functions, together with the thesis title, as a stand-alone text. Abstracts appear, absent the full text of the thesis, in bibliographic indexes such as PsycInfo. They may also be presented in announcements of the thesis examination. Most readers who encounter your abstract in a bibliographic database or receive an email announcing your research presentation will never retrieve the full text or attend the presentation.
% An abstract is not merely an introduction in the sense of a preface, preamble, or advance organizer that prepares the reader for the thesis. In addition to that function, it must be capable of substituting for the whole thesis when there is insufficient time and space for the full text.
% Size and Structure
% Currently, the maximum sizes for abstracts submitted to Canada's National Archive are 150 words (Masters thesis) and 350 words (Doctoral dissertation).
% To preserve visual coherence, you may wish to limit the abstract for your doctoral dissertation to one double-spaced page, about 280 words.
% The structure of the abstract should mirror the structure of the whole thesis, and should represent all its major elements.
% For example, if your thesis has five chapters (introduction, literature review, methodology, results, conclusion), there should be one or more sentences assigned to summarize each chapter.
% Clearly Specify Your Research Questions
% As in the thesis itself, your research questions are critical in ensuring that the abstract is coherent and logically structured. They form the skeleton to which other elements adhere.
% They should be presented near the beginning of the abstract.
% There is only room for one to three questions. If there are more than three major research questions in your thesis, you should consider restructuring them by reducing some to subsidiary status.
% Don't Forget the Results
% The most common error in abstracts is failure to present results.
% The primary function of your thesis (and by extension your abstract) is not to tell readers what you did, it is to tell them what you discovered. Other information, such as the account of your research methods, is needed mainly to back the claims you make about your results.
% Approximately the last half of the abstract should be dedicated to summarizing and interpreting your results.