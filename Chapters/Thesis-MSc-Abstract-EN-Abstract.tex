% #############################################################################
% Abstract Text
% !TEX root = ../main.tex
% #############################################################################
% use \noindent in firts paragraph

\noindent 

%1) Introduction - find the best route, schedule and set of flights for a multi-city trip.
%2) Overview of related problems - gives an idea on what strategies to follow
%3) Formal Definition of the problem and proposed methods to solve it
%4) Designing and implementation a web application for it
%5) Experimental results - optimization system performance, utility of the proposed problem, and comparison to current state of the art.

Despite the existence of numerous flight search applications, most of them lack the ability to properly address multi-city flight requests. The goal of this work is to develop an application that addresses the problem of finding the best route, schedule and set of flights, for any unconstrained multi-city flight request. 

The considered problem belongs to the class of scheduling and routing problems. In particular, it occurs as a generalization of the Traveling Salesman Problem and its time-dependent variation. Thus, an overview of the methodologies implemented in the resolution of these problems may be useful for the definition of an adequate strategy to the resolution of the considered problem. 

This work proposes a formal definition for the problem under resolution, and denotes it as the Flying Tourist Problem. In order to solve it, this work uses different heuristics and metaheuristics, including the Ant Colony Optimization and the Simulated Annealing. Using these methods, a solution may be constructed in real-time, even for large problem instances.

The methods developed for the resolution of this problem are integrated into a web application, making it publicly available.

The developed system is evaluated using three different criteria: the quality of its optimization system; the utility of the devised problem; and its performance compared to other similar systems. The results are extremely positive, and show that it is possible to
considerable reduce the price and flight duration, even for small problem instances. 

Keywords: Flight search application, Routing and scheduling problem, metaheuristics

%reduce the cost in 15\% on a 5 cities trip, and 33\% on a 10 cities trip.

%The performance of the developed optimization methods is evaluated using a set of benchmark tests, while the utility of the proposed system is determined according to the price savings, when compared to the standard way of solving these type of requests. 
%The results presented in this work are extremely positive, showing that is is possible to reduce the cost in 15\% on a 5 cities trip, and 33\% on a 10 cities trip.
%The results presented in this works are extremely positive, and show that, as the number of cities to visit increases, the proposed system is capable of presenting solutions with a lower flight price and duration, when compared to the standart me


%This problem occurs as a generalization of the Traveling Salesman Problem and thus, an overview of this problem and the methodologies used in its resolution may be useful for the definition of an adequate strategy for the resolution of the considered problem.




% How to Write an Abstract for Your Thesis or Dissertation
% What is an Abstract?
% The abstract is an important component of your thesis. Presented at the beginning of the thesis, it is likely the first substantive description of your work read by an external examiner. You should view it as an opportunity to set accurate expectations.
% The abstract is a summary of the whole thesis. It presents all the major elements of your work in a highly condensed form.
% An abstract often functions, together with the thesis title, as a stand-alone text. Abstracts appear, absent the full text of the thesis, in bibliographic indexes such as PsycInfo. They may also be presented in announcements of the thesis examination. Most readers who encounter your abstract in a bibliographic database or receive an email announcing your research presentation will never retrieve the full text or attend the presentation.
% An abstract is not merely an introduction in the sense of a preface, preamble, or advance organizer that prepares the reader for the thesis. In addition to that function, it must be capable of substituting for the whole thesis when there is insufficient time and space for the full text.
% Size and Structure
% Currently, the maximum sizes for abstracts submitted to Canada's National Archive are 150 words (Masters thesis) and 350 words (Doctoral dissertation).
% To preserve visual coherence, you may wish to limit the abstract for your doctoral dissertation to one double-spaced page, about 280 words.
% The structure of the abstract should mirror the structure of the whole thesis, and should represent all its major elements.
% For example, if your thesis has five chapters (introduction, literature review, methodology, results, conclusion), there should be one or more sentences assigned to summarize each chapter.
% Clearly Specify Your Research Questions
% As in the thesis itself, your research questions are critical in ensuring that the abstract is coherent and logically structured. They form the skeleton to which other elements adhere.
% They should be presented near the beginning of the abstract.
% There is only room for one to three questions. If there are more than three major research questions in your thesis, you should consider restructuring them by reducing some to subsidiary status.
% Don't Forget the Results
% The most common error in abstracts is failure to present results.
% The primary function of your thesis (and by extension your abstract) is not to tell readers what you did, it is to tell them what you discovered. Other information, such as the account of your research methods, is needed mainly to back the claims you make about your results.
% Approximately the last half of the abstract should be dedicated to summarizing and interpreting your results.