\subsubsection{Pseudo-random construction procedure}
\label{sec:pseudo_random}

Formally, The method introduced in this subsection is not an optimization algorithm, but rather a solution construction procedure for the Flying Tourist Problem. This procedure is relevant for two reasons. First, it can be used to construct a preliminary solution to a given request in a very fast manner. Naturally, the quality of this solution may be extremely poor, but it is very useful as an initial and fast response to a user request. Furthermore, this construction procedure is also relevant, because some optimization algorithms, like the Simulated Annealing (discussed in section \ref{sec:sa}), require an initial valid and complete solution.

The method introduced below will be hereinafter called pseudo-random construction procedure, and requires an instance of the Flying Tourist Problem $G=(Vc,A,T_0,D,TW)$ where there are no restrictions regarding the time-windows, that is $TW(i) = [0, +\infty[$, $\forall i \in V$. The procedure can be summarized as follows:

\begin{enumerate}
\itemsep0em 
    \item set an initial empty solution: $s=()$;
    \item set the current time to one of the possible start dates: $t \in T_0=[T_{0i}, T_{0f}]$;
    \item set the current node to the start node: $v_c = v_0$;
    \item if the set of nodes to visit, $V$, is empty, go to step 11); else, continue to step 5)
    \item select the next node by choosing a random node from the set of nodes to visit: $v_i \in V$;
    \item remove the selected node from the set of nodes to visit: $V=V\backslash\{v_i\}$;
    \item extend the solution with the arc: $a_{v_c, v_i}^t$;
    \item increment the time according to the duration of visit of the selected node: $t=t+d(v_i)$;
    \item update the current node: $v_c=v_i$;
    \item go to step 4);
    \item extend the solution with the final arc $a_{v_c, v_{n+1}}^t$.
\end{enumerate}

The proposed pseudo-random construction procedure is expected to be adequate for single-flights and round-trips, as well as small multi-city requests. As the number of cities to visit increases, the quality of the solutions presented by this solutions is expected to fall. In any case, despite the size of the instance under resolution, this procedure is expected to be very fast and able to return a solution to a request in a very short time.