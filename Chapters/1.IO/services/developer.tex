\subsection{Developer search tools}
\label{sec:developer_st}

From the perspective of the developer who intends to create a flight search engine, there are several OTA that provide access to their \ac{API}, allowing the direct access to flight data. These APIs usually offer the possibility of searching for cached and, sometimes, real-time flight data. In some cases, these APIs also extend their range of services, and include endpoints for the access of hotel information, car-rental and railroad services, and even cruise ships itineraries \cite{expedia_docs, amadeus}.

An API may be classified as \textit{public}, \textit{limited} or \textit{private} according to the restrictions it imposes to its access. Private APIs are those which are only accessible to enterprises. An example of a company whose API is private is \textit{Skyscanner}. In its turn, \textit{Google} offers a service which can be classified as limited, since it only offers up to 50 free queries per day. Finally, there are other APIs whose access is completely free and unlimited, and therefore, are classified as \textit{public}. \textit{Kiwi} is an example of a company that provides a public API.

Given the academic purpose of this work, the selection of a flight API is restricted to those that are \textit{public}. Unfortunately, the number of public flight APIs is very short. Among the 10 companies enumerated in the previous section, only three have public APIs: \textit{Expedia, Amadeus} and \textit{Kiwi}. The first two companies operate mostly in North America and their flight data is mostly restricted to that continent. In its turn, \textit{Kiwi} is an European company, but does not limit its services to this continent. Instead, its API operates as a meta-search engine, aggregating data from different content providers. 

As a consequence, during the development of this work, the necessary flight data will be provided by the \textit{Kiwi} public API. The major search utilities that this particular API offers are \cite{kiwi_api}:

\begin{itemize}
    \item access to relevant information about cities, airports and airline companies;
    \item access to single-flight and round-trip information, with flexible queries that include:
        \begin{itemize}
            \item flexible start dates;
            \item flexible durations of stay;
            \item queries with undefined destination;
            \item queries with multiple origins and destinations;
        \end{itemize}
  \item the possibility to aggregate up to 9 single-flights and round-trip queries in a single request.
\end{itemize}

It is worth noting that, although Kiwi (and other companies) enable the query of multiple flights at once, each flight must specify a particular pair of cities and date. Thus, this corresponds to a constrained multi-city search, as discussed in the previous section, and does not actually provide information regarding the best route, schedule or set of flights for an unconstrained multi-city trip.  
