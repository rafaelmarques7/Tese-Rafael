The tourism industry dates to the 19th century, but it was impacted by some significant chapters of human technology, which lead to an increased and sustained growth of its market size. First, during the 1920's, the development of commercial aviation had a significant positive impact on the industry, shifting the transportation focus to the airplane. Much later, during the 90's, the establishment of the internet led to some changes in the market, because airlines could sell directly to the passengers \cite{tourism_tec}. More recently, the widespread use of mobile phones lead to a new increase in the markets size. In 2016, the direct contribution of the tourism industry for the GDP was over 2.3 trillion dollars, while the total contribution was over 7.6 trillion dollars\cite{travel_report}.

The market size growth of the tourism industry is sustained by traveling agencies, whose main function is to serve as an agent, advertising and selling products and services on behalf of other \cite{book_tourism}. These services usually include, but are not limited to, transportation, accommodation, insurance, tours and other tourism associated products. In recent years, Online Travel Agencies (OTA) became particularly important to the industry. This is because they allow a fast, direct and rich interaction with the user. Using only a cell-phone, most people can, in a manner of minutes, search and book a flight, hotel services, and if necessary, even car rental. 
   
Most OTA function as a meta-search engine, that is, they perform a search across multiple independent travel services providers. This is a significant difference to the services provided by Direct Travel Supplier (DTS), that are limited to display the results of their own services. Examples of DTS are airlines, hotels and car rental companies which, usually, sell their own product or service directly to the client. On the other hand, OTA usually do not own any travel services, but serve solely as an intermediary between the traveler and the travel services provider. Recent reports show that OTA are increasing their market share, but direct travel suppliers still account for 57\% of the total online travel consumption \cite{OTA_industry_report}.

% In order to better understand the difference between OTA's and direct travel suppliers, 
% consider the case of a user searching for a simple round flight between two cities. 
% If this user visits an individual airline website, the flight results presented are limited to those offered by this airline.
% However, there is no guarantee that the airline flies the user defined route.
% Furthermore, even if there is such a route, it is possible that this route is not flown every day.
% Note that these two types of problems are very common, especially in low-cost airlines.
% In contrast, the same round flight search on a metasearch engine of some OTA 
% will produce a variety of results which include several different airlines.
% Finally, since OTA aggregate data from different airlines and other meta searches,
% it is less likely for the two problems described above to occur using the OTA search tools.
% In conclusion, collecting data from multiple sources usually results in a higher variety and quality of results.

The focus of this section will be given to the overview of the search tools provided by different OTA upon the request of information regarding a single-flight, as well as round and multi-city trips. 
%This overview is made from both the user and the developer point of view, because the type of services provided by OTA may vary according to this. 
Because the search tools provided by OTA may vary according to the entity making the request, this overview is done both from the user and the developer point of view, in sections \ref{sec:user_st} and \ref{sec:developer_st}, respectively.

%Despite the variety of services provided by OTA, this section will focus on making a review of the existing search tools available for the commercial flight transportation segment. This analysis will be made from both the user and the developer point of view
%because the offered services often vary according to this. 

%While