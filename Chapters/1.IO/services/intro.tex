The tourism industry dates to the 19th century, but it was impacted by some significant chapters of human technology, which led to an increased and sustained growth of its market size. First, during the 1920's, the development of commercial aviation had a significant positive impact on the industry, shifting the transportation focus to the airplane. Much later, during the 90's, the establishment of the internet led to some changes in the market, because airlines could sell directly to the passengers \cite{tourism_tec}. More recently, the widespread use of mobile phones led to a new increase of the markets size. In 2016, the direct contribution of the tourism industry for the GDP was over 2.3 trillion dollars, while the total contribution was over 7.6 trillion dollars\cite{travel_report}.

The market size growth of the tourism industry is sustained by traveling agencies, whose main function is to serve as an agent, advertising and selling products and services on behalf of others \cite{book_tourism}. These services usually include, but are not limited to, transportation, accommodation, insurance, tours and other tourism associated products. In recent years, Online Travel Agencies (OTA) became particularly important to the industry. This is because they allow a fast, direct and rich interaction with the user. By using only a cell-phone, most people can, in a matter of minutes, search and book a flight, hotel services, and, if necessary, even car rental. 
   
Most OTA operate as meta-search engines, that is, they perform a search across multiple independent travel services providers. This is a significant difference to the services provided by \ac{DTS}, which are limited to offer their own services. Examples of DTS are airlines, hotels and car rental companies which, usually, only sell their own product or service directly to the client. On the other hand, OTA usually do not own any travel services, but serve solely as an intermediary between the traveler and the travel services provider. Recent reports show that OTA are increasing their market share, but direct travel suppliers still account for 57\% of the total online travel consumption \cite{OTA_industry_report}.

Hence, this section will present a brief overview of the search tools provided by different OTA upon the request of information regarding a single-flight, as well as round and multi-city trips. Since the search tools provided by OTA may vary according to the entity making the request, this overview is done both from the user and the developer point of view, in sections \ref{sec:user_st} and \ref{sec:developer_st}, respectively.