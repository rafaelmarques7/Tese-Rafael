\subsection{Mobile applications}

During the decade of 2010, web applications have evolved in various ways,
and acessing the web via mobile phone is now a reality. More than that,
the mobile web usage has overtaken the desktop usage. This means that
when building web applications, the design process has to account for
both mobile and desktop usage.

When building a mobile application, there are currently two valid options.
The first is writing native apps, that is, an application that has been developed
for use on a particular platform or device. When building native apps,
programmers usualy need to build different programs for different devices.
These programs are usualy written in a diferent language, and have diferent underlying technologies,
while the overal functionality presented to the user is, generaly, the same.
Native apps are usualy writen in diferent programming languages, depending on the
usage device. Android apps are usualy writen in Java or C/C++, while iOS apps
are usualy writen in Objective-C or Swift.

The second option when writing mobile applications is choosing a framework
which produces an aplication that can run on any device. This is the case of React-Native.
React-Native is a javascript library which allows the production of mobile apps
using only Javascript. The developed apps are native apps,
indistinguishable from an app built using Objective-C or Java.

The main advantage of React-Native is that by building one application,
it will be available for any device, may it be Android, iOS or Windows mobile.
This reduces the workload when building a mobile application. Instead of having
to code diferent applications for different devices, using React-Native,
there is only need to build one application, which will then be available to all devices.
Furthermore, this also reduces the strugles of having to maintaining
and updating different mobile applications.

React-Native has an additional advantage worth mentioning. React-Native and React
are two javascript libraries for developing mobile and web applications, respectively.
These two libraries are very similar, and the majority of the changes are relative
to the tags used to represent elements of the DOM. While web applications use the traditional
HTML tags, mobile applications have associated different tags. For example, the 'div' tag
in HTML is replaced by the 'View' tag in React-Native. The underlying diferences between
both libraries are minimal, and thus, writing web applications and mobile applications
with these two frameworks share the same overal logic, while difering in a few lines of code.
