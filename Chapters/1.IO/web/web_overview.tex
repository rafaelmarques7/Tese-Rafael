\subsection{Web overview}

The WorldWideWeb is a project invented and developed by Tim Berners-Lee in CERN
during the year of 1989, and released to the public in 1991. As described by
Tim himself in an internal CERN e-mail, later released to the public,
the WWW is a project intended to be a "powerful global information system",
to spread "information freely to anyone". The WWW model intends to overcome
the incompatibilities of data format, by having the data handled by
"smart browser and smart server", which are responsible for presenting it to the user.

The web consists of documents and links,
which point to other documents or to places within documents. Links
can be followed by clicking on them, which result in the display of a new document.

Usualy, the web is acessed via web browser. The browser is a program
which retrieves information resources and renders
them to allow visualization. Although the browsers main function is retrieving and displaying
information from the web, it can also be used to display local files.

On the web, the information exchange is processed by a particular protocol. The
most widely used is the HyperText Transfer Protocol, HTTP, althoug the web supports
a few others, like the File Transfer Protocol, FTP.
HTTP is a stateless request-response protocol for hypermedia information systems.
The protocol is designed for the transportation of data from one network point
to another. These points usualy represent clients or web servers.

The initial HTTP protocol had only one method, GET. This was because the web
was designed to be used as a unidirectional flow of information.
Users requested an external resource, via HTTP GET, and the server
retrieves the resource without any action required from the external party.

To request a resource, there are two possible ways. The first, is by clicking on a
hypertext link, which points to another document, by having an associated network address.
The second is by requesting a resource via Uniform Resource Locatorn, URL.
The prefix of the URL, known as the Uniform Resource Indentifier, URI, determines
how the URL is interpreted. The URI indicates the protocol used, as well as a hostname
that is associated with a network address. The URL can be further completed with
specific resources or queries.

To overcome the incompatibilities of data format, the introduction of
the web was accompanied by the introduction of the HyperText Markup Language, HTML.
HTML is a way of structuring data in a hypertext document, and it describes
the structure of a web page. Due to its stucture, HTML documents may be seen as trees objects,
where each node represents part of the documents, and where nodes have hierarquichal relationships.
This representation of the HTML document is known as Document Object Model, DOM.

Upon the invention of the web, the way of showing web pages was to render the
elements described in an HTML documents. This document was the only source
of information, and it contained all the necessary information to be rendered by a web browser
and displayed to the user. With the evolution of the web, this is no longer true.
The HTML document is no longer the only necessary entity to "correctly" display a web page.
Today, the technologies used in the rendering of a web page include Cascade Style Sheets, CSS,
and JavaScript, and together with HTML they form the triade for rendering of web pages.
CSS is a way of describing graphic information of a web page, while Javascript is a scripting
language which enables the modification of elements of the web page.

"The World Wide Web Consortium, W3C, is an international community that develops
open standarts to ensure the long growth term of the Web."
