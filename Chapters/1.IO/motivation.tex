\section{Motivation}

The motivation for this work is the resolution of a consumer problem,
felt in first hand. Specifically, how online traveling agencies do not offer
support and optimization for complex airplane routes. In this work, we wish
to address this problem. We propose a complete system for the implementation
of such functionalities in a web application,
aswel as an optimization algorithm for the resolution of such problem.

Online Traveling Agencies are online applications which sell traveling goods,
as commercial flights. Althoug users still have the option of buying
their airplane tickets in the respective airline company website,
57% of today's users buy tickets from an OTA. This businness is increasing in size,
offering a variety of consumer goods, including hotel and car rental,
as well as cruiseships, and are also expecting to enter the rail businness.
In fact, the size of the OTA market is expected to reach a global value of 1.000B\$,
by 2022.  This market is dominated by two comapny aggregates,
while Google, with its Google Flights service, is growing in market size.

The user functionality of the majority of the websites has increased,
by prodiving users with some refined search experience. These functionalities
include price overview over extended search periods
and price tracking in real time.

However, when searching for a trip
with more than two flights, users are not offered any search functionalities.
Instead, to find the cheapest trip, users need to check any
possible combination of flights, one by one. As the trip size increases, finding
the cheapest trip becomes virtualy impossible for a human. Because of this,
users will ultimately pay more than necessary for this service.

Most user trips can be modulated as a Traveling Salesman problem, or a variation
of it. For example, a two way flight is a two city TSP problem. This problem has
a specific depot, and the cost between the two arcs varies over time. This means
that in fact, this problem must be addressed as time dependent TSP variation.

Ant colony optimization is a meta heuristic algorithm which provide
high quality solutions in a short ammount of time. This algorithm
can be adapted to the resolution of any combinatorial optimization Problem,
as is the TSP and the TDTSP.

The Traveling Salesman Problem requires a cost matrix with the
cost of any possible flight between two cities. Because of this,
the resolution of this problem requires an extensive ammount of data.

Public flight API's charge a fixed fee for the request of any flight data.
Due to this, requesting flight information through a public API to describe
a TSP instance is an expensive process. The alternative to this is designing
a data mining algorithm which scrapes the web for the data necessary to solve
the TSP instances.


\subsection{Direct TSP applications}

The TSP is conceived as very simple routing problem, with very few constraints.
Due to its nature, the TSP arises as a subproblem for many transportation
and logistics applications. In fact, the problem of optimizing a school bus route 
had a particular impact in the history of the TSP, as it served as motivation 
for one of the pioneers in the research of the TSP, Merrill Flood.  
(cite $http://www.cs.yorku.ca/~aaw/Zambito/TSP_Survey.pdf$)
 
 During the 1930's and 1940's, a statistical study of aggriculture production 
 was conducted in India, where are the time jute crops accounted for a quarter of the 
 total exports. During this period, Mahalanobis notices that a major component
 in the costs of the survey would be the necessary time to move men and equipment 
 from one sample area to another. In sum, the formulated problem was identical to the TSP.
 Although Mahalanobis does not propose a way for solving the problem,
 a statistical analysis to estimate the lower bound was conducted. 
(cite https://www.isical.ac.in/events/In%20Pursit%20of%20the%20Travelling%20Salesman.pdf)

The TSP has also direct applications in the cost minimization when mapping celestial objects.
The Starlight Interferometer Program is NASA program to map celestial objectives
with high accuracy. This program utilizes a Lin-Kerninghan heuristic to select 
the sequence of celestial objects to be mapped. The objective is to  reduce 
the total fuel spent by repositioning the satellites involved in the interferometer process.
This problem is trated as a TSP. 
(cite: $http://citeseerx.ist.psu.edu/viewdoc/download?doi=10.1.1.40.1906\&rep=rep1\&type=pdf$)


In the manufacturing industry, the TSP has several applications.
It was used to organize computer wiring (cite: Lenstra $\&$ Rinnooy Kan, 1974).
Also related to computer wiring are, more recently, the circuit board  manufacturing processes
(cite: $http://ieeexplore.ieee.org/document/1205218/?reload=true and http://digital-library.theiet.org/content/conferences/10.1049/cp_19991183$),
which can be modulated as a TSP.
Finally, a classic application of the TSP is the drilling problem (cite: $https://link.springer.com/content/pdf/10.1007/BF01415960.pdf$). Actually,
due to the diferent sizes of the drills, 
the drilling problem may be viewed as a series of TSP's. 
The objective is to minimize the total time required in the manufacturing of a 
printed circuit board (PCB's), maximizing thus the throughoutput.
%(sourcE: $http://www.exatas.ufpr.br/portal/docs_degraf/paulo/Traveling_Salesman_Problem__Theory_and_Applications.pdf$)
(NOTE: We can easily expend this paragraph if we want to.)
 
 
 
 
\subsection{Tsp related problems}

The Traveling Salesman Problem is a densily researched problem because it is 
one of the simples routing problem, yet, solving it may be a very difficult task. 
Many routing problems are generalization of the simple TSP. An example of this is the 
is the Vehicle Routing problem. It can be stated in one line as "What is the optimal
set of routes for a vehicle fleet to serve a set of costumers?". The VRP requires 
each costumer to be visited once and once only, by a single vehicle. The objective 
is to find a set of routes which optimizes some decision variable, as cost or serving time.
If there is only one available vehicle, the VHR becomes the simple TSP. 

When organizing a routing problem, there are many real life limitations that may be imposed on the problem.
When handling with customers, mosten, there is a timetable which has to be respected. 
This leads to a series of time constrained VRP. (cite TW VRP...)
When handling with cargo distribution, the weights or dimensions of each cargo unit 
can also be taken into account when designing the problem. This class of problems 
is known as the capacited VRP (cite CVRP).

INTRODUCE TIME DEPENDENCY 
 









