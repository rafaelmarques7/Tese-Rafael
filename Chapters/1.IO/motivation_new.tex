\section{Motivation}

Online Traveling Agencies (OTA) are online applications that sell traveling goods, as, for example, commercial flight tickets. Although many consumers retain the option to buy flights directly from airline companies, the majority opts to use OTA. The main reason for this is that these agencies aggregate flight data from multiple airlines, instead of being limited to a single one, which ultimately increases the options of the consumer. Furthermore, many OTA work as meta-search engines, searching over a variety of websites in order to find the best flights which satisfy the consumer requirements. However, while OTA usually provide very complete search functionalities for simple flights, the majority fails to offer the same search options for a trip composed of multiple cities. 

As an example, consider a trip starting and ending at a given city, which must visit every other city specified in a particular list of cities. If there are no constraints associated with the order in which these cities must be visited, this problem is already well known in the scientific community as the Traveling Salesman Problem (TSP). This problem is considered very difficult to solve, since the total number  of possible solutions increases in an exponential way, as the number of cities increases.

However, when commercial flights are the means of transportation between every pair of cities, this problem can no longer be considered the \textit{classic} Traveling Salesman,but rather its \textit{time-dependent} counterpart. This is because some of the major flight characteristics, as its price and duration, cannot be considered constant over time. Rather, they are dependent on the particular flight that is selected, and the characteristics of that flight may follow no apparent logical rule, at least from the consumers perspective.

Consequently, finding the most efficient set of flights, from the consumer point of view, tends to be a repetitive and time-consuming task. Faced with this problem, only the most persistent consumer will be able to find the best solution for the problem, and this can only  occur for a very small list of cities. As the number of cities increases, even the most persistent consumer can hardly verify all possible solutions. This means that, ultimately, the final consumer will pay more than necessary for the requested service.

This thesis also arises as a response to the public contest called \textit{Traveling Salesman Challenge} \cite{tsc_kiwi}, issued by \textit{Kiwi}, a well established OTA, even though the beginning of this work dates prior to the issue of the challenge. In this challenge, Kiwi recognizes that, in most cases, users do not care about the order in which they visit a given list of cities.
The challenge was set since most OTA do not offer these type of search tools, due to the computational complexity associated with the problem, although there exists a market niche with interest in these type of services.
%and that there exists a market niche interested in this type of services. Kiwi also recognizes that most OTA do not offer these type of services due to the computational 
%complexity associated with the problem.

In accordance, this work intends to address the problem of solving unconstrained multi-city trips, by studying it and by developing the necessary tools to effectively solve the problem in a time-efficient manner. It also aims at the development of a proof of concept online flight search application, 
implementing these technologies in order to, ultimately, provide high-quality search
for complex flight requests.

